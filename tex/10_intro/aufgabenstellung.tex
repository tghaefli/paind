Diese Arbeit handelt von einer bidirektionalen Datenstromrealisierung zwischen Simulink und einem embedded Board namens Pixhawk. Diese Daten sollen in einer Hardware in the Loop Simulation verwendet werden.\\

\noindent Im Kapitel \ref{sec:Was ist Pixhawk} und \ref{sec:Toolchains} werden die Grundlagen von Pixhawk sowie der Programmierumgebung erklärt und aufgezeigt. Weiter wird das Konzept der Hardware in the Loop Simulation an einigen Beispielen veranschaulicht, welches anschlissend mit dem Pilot Support Package versucht wurde zu realisieren.\\

\noindent Als Lösungskonzept der Hardware in the Loop Simulation wurde schlussendlich der Programmcode vom Pixhawk geändert. Auf der Gegenseite wurde im Simulink eine Datenstromverarbeitung und Simulation erstellt. Dieser Ansatz wird im Kapitel \ref{sec:App entwicklung} sowie \ref{sec:Simulink} aufgezeigt.



\subsection{Aufgabenstellung}

Bei der folgenden Aufgabenstellung handelt es sich um eine Kurzzusammenfassung. Die komplette Anforderungsliste ist in Kapitel \ref{sec:Anhang} ersichtlich.

\begin{itemize}
\item Einarbeitung in die Pixhawk Firmware und Designpatterns
\item Eigene Test-App mit Datenstromverarbeitung demonstrieren
\item Kommunikationsmöglichkeiten zwischen Pixhawk und Simulink ausarbeiten
\item Hardware in the Loop Simulation verwirklichen
\item Eigene App schreiben, welche die Komunikation mit Simulink übernimmt
\item Einfache Hardware in the Loop Simulation auf Seiten Simulink programmieren und demonstrieren.
\end{itemize}

\clearpage