\section{Auswertung}

\subsection*{Fachliches Fazit}
\noindent 
Durch die Pixhawk App wurde ein mächtiges Tool entwickelt, um Daten auszulesen, zu modifizieren und einzuspeisen. Dieses kann auch als Basis für andere Aufgaben dienen. Selbes gilt auch für die Simulink Blöcke. \\
\noindent
Durch das Zusammenspiel der Pixhawk App sowie dem Simulinkmodell wurde erfolgreich eine stabile Datenstromverarbeitung realisiert. Die Daten können auf der jeweiligen Seite mit einer hohen Baud gesendet, empfangen und interpretiert werden. Die CPU Auslastung ist auf beiden Seiten sehr gering in Anbetracht der grossen Datenmengen.\\
Eine HiL Simulation konnte in der vorgegebenen Zeit nicht realisiert werden.

\subsection*{Persönliches Fazit}
Durch dieses spannende Arbeit erhielt ich Einblicke in die Aviatik, Flugregelung, C sowie C++ Programmierung
Die Arbeit war in drei Aufgaben aufgeteilt und jede Arbeit erforderte unterschiedliche Tools. Zum einen musste man sich in das PSP und den Code Composer einarbeiten, zu anderen in die Pixhawk Firmware und zum Schluss noch ins Simulink mit einer Datenstrom-Verarbeitung und -Ausgabe. Jedes dieser Tools benötigte eine Einarbeitungszeit. Das erlernte Hochschul-Wissen konnte teilweise angewendet werden, jedoch haben Elektroniker ein sehr kleines Programmierwissen im Vergleich zu den Informatikern. Aus meiner Sicht hätte ein Interdisziplinäres Team zu einem besseren Ergebnis geführt.\\
Durch die Einzelarbeit konnte ich jedoch einen grösseren, persönlichen Nutzen erarbeiten. Es ermöglichte einen Einblick in die anderen Fachgebiete.

\clearpage
