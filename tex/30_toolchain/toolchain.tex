\subsection{Installation}
\begin{flushleft}

Die Installation der Pixhawk Toolchain wird auf \href{https://pixhawk.org/dev/quickstart}{www.pixhawk.org/dev/quickstart} für alle Betriebssysteme sehr detailliert und ausführlich erklärt.

\end{flushleft}



\subsection{Verwendung}
\noindent
Mit dem Kommando \textbf{make px4fmu-v2\_default} wird die Firmware kompiliert und die .elf Datei erzeugt. Mit \textbf{make px4fmu-v2\_default upload} wird die Firmware kompiliert, .elf Datei erzeugt, auf das Pixhawk per USB hochgeladen und dort programmiert.

\subsubsection{Linux}

Für Linux Benützer ist es zu empfehlen, ein alias für die Kommandos zu erstellen. Dies kann erzeugt werden durch:
\begin{lstlisting}
cd ~
vim .bashrc
\end{lstlisting}

\noindent
Anschliessend die Taste 'a' drücken für den Eingabe Modus. Jetzt folgende Zeilen eingeben:

\begin{lstlisting}
mk () {
  cd ~/path/to/Firmware/
  make px4fmu-v2_default
}

mkup () {
  cd ~/path/to/Firmware/
  make px4fmu-v2_default upload
}
\end{lstlisting}

\noindent
Der Pfad zur Firmware muss vorher angepasst werden. Durch ESC kann anschliessend in den Navigationsmodus gewechselt werden. Mit ':wq' werden die Änderungen gespeichert und das Programm verlassen. Der Pfad zur Firmware muss vorher angepasst werden. Zum Schluss muss die  \textbf{.bashrc} Datei neu kompiliert werden mit:

\begin{lstlisting}
source .bashrc
\end{lstlisting}

\noindent
Nun kann über das Terminal die Firmware kompiliert und hochgeladen werden mit dem Befehl 'mkup'. Falls man diesen nicht immer neu eingeben möchte, kann mit der Pfeiltaste $\uparrow$  oder !! das letzte Kommando erneut ausgeführt werden.\\

\noindent
Für die Programmierung empfiehlt sich die IDE Code::Blocks. Auch hierzu gibt es eine Anleitung unter \href{https://pixhawk.ethz.ch/toolchain/codeblocks}{https://pixhawk.ethz.ch/toolchain/codeblocks}. In der Anleitung sollten jedoch nur die stable builds verwendet werden und keine nightly builds.


\subsection{Troubelshooting}
Während der Arbeit mit dem Pixhawk traten keine Fehler auf. Das px4 v1 (Pixhawk Version1) hatte im Vergleich einige zeitkritische Probleme beim Löschen und Programmieren der Firmware. 
\clearpage
