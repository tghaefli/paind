\subsection{uORB}

Die uORB (\acrlong{uorb}) wird auf dem Pixhawk verwendet um Datenstrukturen, sogenannte Topics, zwischen Apps, also Threads, auszutauschen. Dies ermöglicht eine ressourcenarme Möglichkeit, auf interne Daten zu warten per Betriebssystem Interrupt.
Man kann einen Filedescriptor auf ein gewünschte Topic anlegen, wie in folgendem Codebeispiel ersichtlich ist. 

\begin{lstlisting}
  //File descriptor erzeugen und zuweisen
	struct sensor_combined_s raw;
	int sensor_sub_fd = orb_subscribe(ORB_ID(sensor_combined));
	struct pollfd fds_uorb[]= {
		{ .fd = sensor_sub_fd,   .events = POLLIN },
	};
	
    [...]
  
	//Warte auf neue Daten per OS Interrupt
	poll_ret = poll(fds_uorb, 1, TIMEOUT_MS);
	
	if (poll_ret <= 0)
	  //Keine Daten erhalten
	else
	  //Neue Daten erhalten

\end{lstlisting}

%\caption{Thema abonnieren und auf Daten warten}

\noindent
Hier wird ein Filedescriptor erzeugt, welcher alle Sensordaten abonniert. Falls während dem Timeout 'TIMEOUT\_MS' neue Daten auf die uORB geschrieben wurden, wird ein Interrupt ausgelöst und der return Wert wird grösser 0 sein.

